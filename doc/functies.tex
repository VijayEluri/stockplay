% Verkleinde margin entry
%\setlength{\marginparwidth}{1.2in}
%\let\oldmarginpar\marginpar
%\renewcommand\marginpar[1] {\-\oldmarginpar[\raggedleft\footnotesize #1]%
%{\raggedright\footnotesize #1}}

% Een TODO-entry
\newcommand{\todo}[1] {
%	\addcontentsline{tdo}{todo}{\protect{#1}}
%	\marginpar{#1}
}

% Een lijst van TODO-entries
\makeatletter
\newcommand \listoftodos {
	\section*{Todo list} \@starttoc{tdo}
}
\newcommand\l@todo[2] {
	\par\noindent \textit{#2}, \parbox{10cm}{#1}\par
} \makeatother

% Float defini\"eren voor codefragmenten
\usepackage{float}
\floatstyle{ruled}
\newfloat{code}{thp}{lop}
\floatname{code}{Codefragment}

% Hyperlink maken en URL in footnote tonen
\usepackage{hyperref}
\newcommand{\makeurl}[2]{\href{#2}{#1} \footnote{#2}}

% Functiedefinitie voor protocolstudie
\newcommand{\function}[5] {
	\subsubsection*{#1}
	\begin{tabular}{|r p{11cm}|}
	\hline
	\textsc{Gebruik} |		& #2 \\
	\textsc{Parameters} |		& #3 \\
	\textsc{Output} |		& #4 \\
	\textsc{Autorisatie} |		& #5 \\
	\hline
	\end{tabular}
}

% Functiedefinities voor logboek handling
\usepackage{ifthen}
\newcommand{\lbdate}{}
\newcommand{\lbsetdate}[1]{
  \gdef\lbdate{#1}
}
\newcommand{\lbentry}[5] {
	\ifthenelse{\equal{\lbdate}{#1}}
	{
	}
	{
		\ifthenelse{\equal{\lbdate}{}}{}{
			& & & \\ \hline % TODO: workaround, zou niet nodig moeten zijn
			\end{tabular}
		}
		\subsection*{#1}
		\begin{tabular}{|r r r p{10cm}|}
		\hline
		\textsc{Begin} & \textsc{Einde} & \textsc{Duur} & \textsc{Beschrijving} \\
		\hline
	}
	\lbsetdate{#1}
	#2 & #3 & #4 & #5 \\
}
\newcommand{\lbstop}[1] {
	& & & \\ \hline % TODO: consistentie fix, zie hierboven
	\end{tabular}
	\lbsetdate{}
	
	\subsection*{Sommatie}
	\textbf{Totale aantal werkuren}: #1.
}

% Compacte enumeraties
\newenvironment{enumerate_compact}{
\begin{enumerate}
  \setlength{\itemsep}{1pt}
  \setlength{\parskip}{0pt}
  \setlength{\parsep}{0pt}
}{\end{enumerate}}
\newenvironment{itemize_compact}{
\begin{itemize}
  \setlength{\itemsep}{1pt}
  \setlength{\parskip}{0pt}
  \setlength{\parsep}{0pt}
}{\end{itemize}}

% Compacte environment voor use-cass
\newenvironment{compact}{\setlength{\parskip}{0pt}}{}

%stijlen voor listings
%
\lstdefinestyle{SQL}{
  breaklines=true,
  language=SQL,
  basicstyle=\normalsize,
  keywordstyle=\ttfamily\color{OliveGreen},
  identifierstyle=\ttfamily\color{CadetBlue}\bfseries,
  commentstyle=\color{Brown},
  stringstyle=\ttfamily,
  showstringspaces=true
}


\definecolor{myid}{rgb}{0.1,0.1,0.1}
\lstdefinestyle{Java}{
language=java,
basicstyle=\normalsize,
numbers=left,stepnumber=1,numberstyle=\small\ttfamily,
numbersep=5pt,frame=tlbr,extendedchars=true,
commentstyle=\color{OliveGreen}\ttfamily,
%% stringstyle=\color{red}\ttfamily,
stringstyle=\ttfamily\color{Magenta},
keywordstyle=\ttfamily\color{Violet}\bfseries,
ndkeywordstyle=\ttfamily\color{Yellow}\bfseries,
identifierstyle=\ttfamily\color{myid},
% sensitive=false,
}
