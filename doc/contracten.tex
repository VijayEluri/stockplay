Beide contracten zijn overeengekomen tussen:
\begin{enumerate}
\item{Vop Hogent (hierna de ``opdrachtgevers''), vertegenwoordigd door:
\begin{itemize_compact}
\item Helga Naessens
\item Ann Vanoverberge
\item Koen Van De Wiele
\item Tim De Pauw
\item Wijnand Schepens
\item Wim Van den Breen
\end{itemize_compact} }
\item{Hogent-iii-ldtt (hierna de ``uitvoerders''), vertegenwoordigd door:
\begin{itemize_compact}
\item Tim Besard
\item Dieter Deforce
\item Laurens Van Acker
\item Thijs Walcarius
\end{itemize_compact} }
\end{enumerate}


%
% Demo
%

\section{Demo contract}

De uitvoerders verbinden zich ertoe uiterlijk tegen 23 maart 2010 een prototype op te zetten van project ''Stockplay''. Dit prototype zal een eerste versie bevatten van het project bestaande uit volgende delen:
\begin{itemize}
\item Een \emph{functionerende en ingestelde databank} draaiende op Oracle software
\item Een \emph{gedeelte van de backend}, bestaande uit minstens het deel wat het aandelenbeheer verzorgt.
\item Een \emph{detailpagina op een ASP.NET website} met daarin het detail van een effect
\item Een eerste start van de \emph{desktop applicatie} die zal instaan voor het beheer van het spel
\item Een eerste versie van een \emph{mobiele applicatie} die reeds kan opstarten op een mobiel toestel
\item Een \emph{voorbeeldpagina met daarin een grafiek} waarin men dynamisch kan inzoomen met behulp van een randapparaat zoals de computermuis.
\end{itemize}

De opdrachtgevers verbinden zich ertoe alle vereisten van het project en het eerste prototype vooraf kenbaar te maken aan de uitvoerders.

Opgemaakt op 26 februari 2010, te Gent


%
% Finaal
%

\section{Definitief contract}

De uitvoerders verbinden zich ertoe uiterlijk tegen 18 mei 2010 een implementatie af te leveren van het project ``Stockplay''. Dit is een een volwaardige gedistribueerde applicatie met een databank, een web- en een desktop-interface. De details van dit project zijn beschreven in de bijgevoegde behoefteanalyse. In grote lijnen zal dit project bestaan uit de volgende delen:
\begin{itemize}
\item Een \emph{functionerende relationele databank} draaiende op Oracle software. Deze bevat verschillende N op N relaties en overervingen. Bij het ontwerp wordt nagedacht over dataredundantie en performance. Er wordt door de uitvoerders rekening gehouden met de 5Gb limiet in de express versie van Oracle. Alsook zal deze enkele cardinaliteitsvoorwaarden bevatten.
\item Een \emph{delegerende backend}, geschreven in de Java programmeertaal. Deze backend zal zeer modulair geschreven worden en kan geraadpleegd worden vanuit elke programmeeromgeving door de abstracte en compacte XML-RPC tussenlaag. Deze backend maakt gebruikt van verschillende design patterns. Er is een losse koppeling tussen de verschillende lagen van de backend. Deze zal geschreven worden met het oog op robuustheid, performantie en flexibiliteit.
\item Een \emph{in ASP.NET C\# geschreven website} met fraaie grafische vormgeving. Deze zal een extra interactieve voeling geven door de ondersteunende JavaScript codes. Bepaalde interactieve delen zullen gebruik maken van de AJAX technologie. De website maakt gebruik van de Model-View-Controller design patterns.
\item Een \emph{desktop applicatie} die zal instaan voor het beheer van het spel. Deze zal in de Java taal geschreven worden. Hierin zal het mogelijk zijn de modules van de spelomgeving te herstarten en te analyseren. Er zullen ook enkele kleine grafieken te zien zijn.
\item Een \emph{mobiele applicatie} die op minstens een van de door ons geteste mobiele toestellen zal werken.
\item De webpagina zal voorzien zijn van \emph{hoogtechnologische grafieken} waarin men dynamisch kan inzoomen met behulp van een randapparaat zoals de computermuis. Alsook zal men het bereik kunnen verplaatsen en beperkt kunnen terugkeren op de handelingen die uitgevoerd zijn.
\item De desktop applicatie zal een \emph{menu} bevatten zoals we deze ook terugvinden in het Windows programma Office Outlook van Microsoft.
\item Er zal in de backend ook een plugin mogelijkheid ingebouwd worden zodat \emph{artifici\"ele speler modules} gebouwd kunnen worden. Enkele van deze modules zullen we ook zelf ontwikkelen en draaien ter demonstratie.
\item Er zal in de backend ook een plugin mogelijkheid ingebouwd worden zodat \emph{extra geavanceerde effectenorders} toegevoegd kunnen worden. Enkele van deze modules zullen we ook zelf ontwikkelen en draaien ter demonstratie.
\item De ontwikkeling zal gebeuren in \emph{verschillende fases}. Na elke fase zullen de uitvoerders aan de opdrachtgevers de nodige documentatie en software afleveren.
\item De uitvoerders zullen gebruiken maken van verschillende \emph{randapparaten} zoals een eID lezer en een mobiele telefoon.
\item De door de uitvoerders geschreven applicatie bevat \emph{verschillende niveaus van authenticatie en autorisatie}. Er is ook een strenge controle daarop in de code.
\item De \emph{foutlogging} zal gebeuren naar een apart bestand of database.
\end{itemize}

Het project zal draaien op enkele publieke servers in beheer van de uitvoerders en op de machines die ter beschikking gesteld zijn door de uitvoerders.
Op de desktop applicatie zal het mogelijk zijn filtering op de dataset door te voeren. De filters op de desktop applicatie zullen door de uitvoerders en de ontwikkelaars vooraf gedefinieerd worden. Op de webapplicatie zal elke mogelijk filter aangemaakt kunnen worden. Er zullen filters aangemaakt kunnen worden die opgebouwd worden met de entiteiten =, <, <=, >, >=, LIKE, AND, OR, NOT. Deze filters kunnen vervolgens dan toegepast worden op alle resultatenlijsten op de webpagina's.

De opdrachtgevers verbinden zich ertoe alle vereisten van het project vooraf kenbaar gemaakt te hebben aan de uitvoerders.

Opgemaakt op 7 maart 2010, te Gent.

