\section{Tim Besard}
\begin{compact}
\lbentry{dinsdag 9 februari 2010}{13:30}{15:30}{02:00}{Initi\"ele briefing over het project en groepsverdeling.}
\lbentry{dinsdag 9 februari 2010}{15:45}{17:45}{02:00}{Kennismaking, brainstorming, verkenning beschikbare databronnen, opstellen draft behoeftenanalyse.}
\lbentry{dinsdag 9 februari 2010}{18:45}{19:30}{00:45}{Proof-of-concept webcrawler voor De Tijd.}
\lbentry{woensdag 10 februari 2010}{17:00}{17:30}{00:30}{Afwerken webcrawler.}
\lbentry{vrijdag 12 februari 2010}{13:30}{17:45}{04:15}{Verdere uitbreiding van de analyse.}
\lbentry{vrijdag 12 februari 2010}{23:00}{23:30}{00:30}{Conversie van Google Doc naar \LaTeX en het maken van Google Code project.}
\lbentry{zaterdag 13 februari 2010}{11:30}{12:00}{00:30}{Herwerken tekst (toevoegen van enkele functies, en een abstract).}
\lbentry{zaterdag 13 februari 2010}{12:15}{12:30}{00:15}{Opzoekingswerk over verschillende codedocumentatiemethoden en conversie naar \LaTeX.}
\lbentry{dinsdag 16 februari 2010}{13:00}{17:45}{04:45}{Afwerken analyse en initieel ontwerp van database en klassenhi\"erarchie.}
\lbentry{woensdag 17 februari 2010}{11:30}{12:30}{01:00}{Begin van XML-RPC backend protocol en reorganisatie van document gebaseerd op cursus Systeemanalyse.}
\lbentry{woensdag 17 februari 2010}{20:30}{21:30}{01:00}{Onderverdelen van backend in verschillende pseudo-klassen en schrijven van initi\"ele documentatie.}
\lbentry{donderdag 18 februari 2010}{10:35}{12:13}{01:38}{Verder werken aan het backend protocol.}
\lbentry{vrijdag 19 februari 2010}{13:30}{17:45}{04:15}{Bespreking en afwerking van het backend-protocol.}
\lbentry{zaterdag 20 februari 2010}{10:00}{10:40}{00:40}{\LaTeX documentatie voorbije week gepusht naar SVN.}
\lbentry{zondag 21 februari 2010}{10:00}{11:00}{01:00}{Afwerken van protocoldefinitie.}
\lbentry{dinsdag 23 februari 2010}{13:30}{14:45}{01:15}{Les in verband met het opstellen van het verslag.}
\lbentry{dinsdag 23 februari 2010}{14:45}{17:45}{03:00}{Opzoekingswerk over de verschillende XML-RPC bibliotheken van Java, en wat afspraken over het gebruik van logging bibliotheken zoals log4j.}
\lbentry{woensdag 24 februari 2010}{15:00}{16:50}{01:50}{Initi\"ele code voor het XML-RPC gedeelte van de backend, en een eerste kijk naar log4j.}
\lbentry{vrijdag 26 februari 2010}{13:30}{17:45}{04:15}{Verdere implementatie van het XML-RPC gedeelte.}
\lbentry{zaterdag 27 februari 2010}{11:00}{12:00}{01:00}{Implementatie van alternatieve ServletServer zodat we interne errors kunnen opvangen, loggen, en ``gereduceerd'' naar de gebruiker te sturen.}
\lbentry{zaterdag 27 februari 2010}{12:00}{12:30}{00:30}{Dummy code die een XML-RPC struct teruggeeft.}
\lbentry{zaterdag 27 februari 2010}{13:00}{13:45}{00:45}{Documenteren van de backendklassen via Doxygen.}
\lbentry{zondag 28 februari 2010}{11:00}{12:00}{01:00}{Makefile toevoegen aan het \LaTeX gedeelte, herorganisatie implementatieklassen, en een eerste versie van de User klasse.}
\lbentry{dinsdag 2 maart 2010}{13:30}{17:45}{04:15}{Verder documenteren van XML-RPC, en de start van een Perl scraper.}
\lbentry{donderdag 4 maart 2010}{17:30}{18:54}{01:24}{Verder afwerken van de Perl scraper.}
\lbentry{vrijdag 5 maart 2010}{13:30}{17:45}{04:15}{Opzet van lokale server (installatie Oracle en Java), plus verbeteren scrape-methodiek scraper.}
\lbentry{dinsdag 9 maart 2010}{13:30}{17:45}{04:15}{Bespreken van demo behoeften, afwerken server, en meer werk aan de scraper.}
\lbentry{dinsdag 9 maart 2010}{18:00}{18:30}{00:30}{Toevoeging van ISIN-nummer aan scraper.}
\lbentry{woensdag 10 maart 2010}{14:30}{15:30}{01:00}{Herwerken van het \LaTeX document.}
\lbentry{woensdag 10 maart 2010}{15:46}{16:13}{00:27}{Business objects integreren in de backend.}
\lbentry{woensdag 10 maart 2010}{16:34}{17:19}{00:45}{Werken aan een dummy implementatie van het backendprotocol.}
\lbentry{woensdag 10 maart 2010}{18:11}{19:16}{01:05}{Design van de filters.}
\lbentry{woensdag 10 maart 2010}{19:18}{19:28}{00:10}{Documenteren van het filterdesign.}
\lbentry{woensdag 10 maart 2010}{20:00}{21:12}{01:12}{Initi\"ele parser voor filters (momenteel enkel de tokenizer) en verder afwerken van POC matching zonder de parser.}
\lbentry{donderdag 11 maart 2010}{12:15}{13:05}{00:50}{Start van parser FSM.}
\lbentry{donderdag 11 maart 2010}{17:53}{20:23}{02:30}{Opnieuw verder werken aan de filter code, nu met een initi\"ele implementatie van een variabele database-connector.}
\lbentry{vrijdag 12 maart 2010}{11:15}{12:56}{01:41}{GraphViz als debug-methode van de filter AST.}
\lbentry{vrijdag 12 maart 2010}{13:30}{15:25}{01:55}{Introductie van Data objecten in de filter.}
\lbentry{vrijdag 12 maart 2010}{15:25}{17:45}{02:20}{Afstellen van filter: toevoegen van datatypes en introduceren van excepties.}
\lbentry{vrijdag 12 maart 2010}{22:00}{22:39}{00:39}{Kleine veranderingen van de filter (herorganisatie, en extra datatypes).}
\lbentry{zaterdag 13 maart 2010}{10:30}{11:19}{00:49}{Parser afgewerkt zodanig dat we ons kunnen richten op belangrijkere zaken. Veel dingen moeten nog toegevoegd worden (relation chaining, l en rvalues), maar het werkt voldoende voor de demo.}
\lbentry{zaterdag 13 maart 2010}{11:46}{12:10}{00:24}{Integratie van filters in andere projecten (backend en BusinessObjects).}
\lbentry{zaterdag 13 maart 2010}{14:40}{16:02}{01:22}{Integratie van filters in DAO laag + interne conversie naar structs ten behoeve van XML-RPC (vermijden van codeduplicatie).}
\lbentry{zondag 14 maart 2010}{09:22}{10:15}{00:53}{Modify routines in DAO laag (fromStruct).}
\lbentry{zondag 14 maart 2010}{10:22}{10:58}{00:36}{Introductie van enkele concrete implementaties in de backend.}
\lbentry{dinsdag 16 maart 2010}{13:30}{17:45}{04:15}{Afwerken van scraper, herwerken van code met als doel een striktere API.}
\lbentry{woensdag 17 maart 2010}{14:46}{15:30}{00:44}{Werken aan scraper, opnieuw re-organisatie gebaseerd op werkelijke beursstructuur.}
\lbentry{woensdag 17 maart 2010}{18:12}{19:53}{01:41}{Parsen via shunting-yard! Algemene methodiek ook een pak verbeterd.}
\lbentry{woensdag 17 maart 2010}{20:00}{20:17}{00:17}{Verbeteren van de signatuur-controle via interface-methoden.}
\lbentry{woensdag 17 maart 2010}{20:20}{20:37}{00:17}{Dynamische instantiatie van operator en functie handler objects.}
\lbentry{woensdag 17 maart 2010}{20:53}{21:33}{00:40}{Implementatie van basis operatorprecedentie (links impliciet).}
\lbentry{donderdag 18 maart 2010}{10:00}{11:45}{01:45}{Realisatieverslag schrijven van de filters.}
\lbentry{donderdag 18 maart 2010}{13:16}{14:24}{01:08}{Parametercontrole via reflectie van statisch methoden (in tegenstelling tot de vorige interface structuur) en constructie van objecten robuuster maken via constructors.}
\lbentry{vrijdag 19 maart 2010}{13:30}{16:00}{02:30}{Afwerken van de scraper -- Parijse beurs toegevoegd en gebruik maken van hard-coded informatie die anders maar moeilijk af te leiden is.}
\lbentry{vrijdag 19 maart 2010}{16:00}{17:45}{01:45}{Documenteren van scraper.}
\lbentry{zaterdag 20 maart 2010}{10:10}{11:10}{01:00}{Wegwerken van de overbodige genericiteit, en bekijken van het singleton probleem.}
\lbentry{zaterdag 20 maart 2010}{10:10}{11:08}{00:58}{Syslog code toevoegen aan de backend.}
\lbentry{zaterdag 20 maart 2010}{13:01}{13:18}{00:00}{Debuggen van een serverside syslog problemen.}
\lbentry{zaterdag 20 maart 2010}{13:43}{15:58}{02:15}{Toevoegen van een syslog webinterface (gebruik makende van phpLogCon) + ``uitbreiding'' van de standaard syslog appender om tagging bug te verhelpen.}
\lbentry{zaterdag 20 maart 2010}{16:33}{17:02}{00:29}{Verplaatsen van libraries, logcode toevoegen aan Parser en implementatie van een initi\"ele FinanceHandler.}
\lbentry{zondag 21 maart 2010}{08:30}{09:17}{00:47}{Backend fixes.}
\lbentry{maandag 22 maart 2010}{16:15}{18:14}{01:59}{Arduino XML-RPC probeersel (blijkt onmogelijk door de SDP limitatie van de bijgeleverde TCP/IP stack).}
\lbentry{vrijdag 26 maart 2010}{13:15}{17:45}{04:30}{Integratie van scraper in backend en fixen van vele kleine bugs.}
\lbentry{vrijdag 26 maart 2010}{20:00}{22:15}{02:15}{Verder afwerken integratie en een oplossing zoeken voor de dateTime.iso8601 limitatie XML::RPC.}
\lbentry{zaterdag 27 maart 2010}{09:50}{10:43}{00:53}{Bugfixes in de backend en scraper, algemene kwaliteit gaat er op vooruit.}
\lbentry{zaterdag 27 maart 2010}{11:56}{12:24}{00:28}{Wegwerken ambigu\"iteiten van constructoren in backend DAO laag.}
\lbentry{zaterdag 27 maart 2010}{13:27}{13:47}{00:20}{Case problematiek weggewerkt, instanti\"eren van objecten verbeterd en robuuster gemaakt.}
\lbentry{zaterdag 27 maart 2010}{13:47}{13:55}{00:08}{Updaten van de testsuite.}
\lbentry{zaterdag 27 maart 2010}{14:10}{15:13}{01:03}{Verbeteren van de manier waarop tijdzones gehanteerd worden binnen de scraper.}
\lbentry{zaterdag 27 maart 2010}{18:47}{20:01}{01:14}{Statistieken toevoegen in de System klasse.}
\lbentry{dinsdag 30 maart 2010}{13:30}{17:45}{04:15}{Bedrijfsbezoek KBC.}
\lbentry{vrijdag 2 april 2010}{13:30}{17:45}{04:15}{Literatuur opzoeken voor implementatie AI, en enig werk aan de scraper.}
\lbentry{vrijdag 2 april 2010}{19:05}{19:45}{00:40}{Lezen van paper (comparing artificial intelligence systems for stock portfolio selection).}
\lbentry{zaterdag 3 april 2010}{14:06}{14:45}{00:39}{Test-suite filters maken.}
\lbentry{zaterdag 3 april 2010}{16:02}{17:29}{01:27}{Lezen van papers (capital market applications of neural networks, fuzzy logic and genetic algorithms en stock market prediction using artifial neural networks).}
\lbentry{zondag 4 april 2010}{16:05}{16:40}{00:35}{Lezen van paper (using neural networks to forecast stock market prices).}
\lbentry{maandag 5 april 2010}{10:35}{11:45}{01:10}{Lezen van lectuur omtrent constructie neurale netten.}
\lbentry{maandag 5 april 2010}{12:25}{14:00}{01:35}{Zoeken van bibliotheek voor gebruik van neurale netten (uiteindelijk gekozen: Flood).}
\lbentry{maandag 5 april 2010}{15:00}{16:24}{01:24}{Neuraal net construeren (input data kiezen, datasets construeren, historische data om te leren opzoeken).}
\lbentry{maandag 5 april 2010}{17:20}{19:10}{01:50}{Zoeken van XMLRPC library om te gebruiken uit C.}
\lbentry{dinsdag 6 april 2010}{11:10}{13:45}{02:35}{Werken aan de scraper (foutafhandeling, indexen, bugfixes, netwerk-optimalisaties, CAC40 en BEL20 toegevoegd, ...)}
\lbentry{dinsdag 6 april 2010}{14:02}{15:06}{01:04}{Migratie van de backend naar het Tomcat framework.}
\lbentry{dinsdag 6 april 2010}{19:00}{19:30}{00:30}{Kleine statistiekenpagina toegevoegd.}
\lbentry{woensdag 7 april 2010}{10:28}{13:25}{02:57}{Implementeren van een Datum object in Parser, schrijven vna een syntaxreferentie voor de filters, toevoegen van een debug pagina, en wegwerken van bugs.}
\lbentry{woensdag 7 april 2010}{14:25}{15:43}{01:18}{One-time scraper voor historische data maken en bulk quote update routines implementeren (factor 15 snelheidswinst).}
\lbentry{woensdag 7 april 2010}{18:16}{20:27}{02:11}{Porten van scraper naar een alternatieve library (RPC::XML) en toevoegen van DAO achtige laag.}
\lbentry{donderdag 8 april 2010}{21:00}{00:00}{03:00}{Deployen van Tomcat.}
\lbentry{vrijdag 9 april 2010}{11:00}{13:00}{02:00}{Server toch omzetten naar Tomcat 6 met de Sun JRE.}
\lbentry{vrijdag 9 april 2010}{14:14}{14:52}{00:38}{Scraper fixes.}
\lbentry{vrijdag 9 april 2010}{15:12}{16:35}{01:23}{Compressie van requests implementeren.}
\lbentry{vrijdag 9 april 2010}{18:22}{19:55}{01:33}{Afwerken compressie van requests.}
\lbentry{maandag 12 april 2010}{09:00}{12:00}{03:00}{Finaliseren deployment backend.}
\lbentry{maandag 12 april 2010}{13:00}{17:00}{04:00}{Afwerken van de scraper.}
\lbentry{maandag 12 april 2010}{19:00}{20:00}{01:00}{Deployment van scraper.}
\lbentry{dinsdag 13 april 2010}{15:37}{16:57}{01:20}{Syslog logging aan de scraper toevoegen.}
\lbentry{dinsdag 13 april 2010}{17:05}{17:36}{00:31}{Datum fixes in scraper en backend.}
\lbentry{woensdag 14 april 2010}{11:00}{12:20}{01:20}{Deployment en bugfixes scraper.}
\lbentry{donderdag 15 april 2010}{10:00}{11:00}{01:00}{Verbeterde foutafhandeling, en verbeteren deployment scraper door er een CPAN compatibele module van te maken.}
\lbentry{donderdag 15 april 2010}{19:30}{21:00}{01:30}{Data visualiseren en bugs fixen}
\lbentry{vrijdag 16 april 2010}{10:00}{11:04}{01:04}{Scraper robuuster maken.}
\lbentry{vrijdag 16 april 2010}{11:54}{13:43}{01:49}{Parser precedentie verbeteren waardoor operatoren makkelijker in gebruik worden.}
\lbentry{vrijdag 16 april 2010}{14:04}{14:52}{00:48}{Bugs in de parser fixen.}
\lbentry{vrijdag 16 april 2010}{14:55}{15:47}{00:52}{Nieuwe filters gebruiken in scraper, enkele bugs fixen, en code robuuster maken voor moest de service wegvallen.}
\lbentry{vrijdag 16 april 2010}{17:49}{19:24}{01:35}{AI implementeren.}
\lbentry{vrijdag 16 april 2010}{20:34}{21:39}{01:05}{AI implementatie afwerken.}
\lbentry{zaterdag 17 april 2010}{10:40}{11:30}{00:50}{PK indexen van autogen-ID naar ISIN veranderen in database en backend.}
\lbentry{zaterdag 17 april 2010}{11:45}{12:11}{00:26}{UserSecurity en IndexSecurity handlers toevoegen.}
\lbentry{zaterdag 17 april 2010}{12:16}{13:08}{00:52}{Indexen toevoegen aan het Perl framework.}
\lbentry{zaterdag 17 april 2010}{13:35}{14:32}{00:57}{Afwerken van het Perl framework, en integreren van de AI.}
\lbentry{zaterdag 17 april 2010}{14:35}{15:15}{00:40}{Werk aan de AI.}
\lbentry{maandag 19 april 2010}{18:46}{20:47}{02:01}{Herstructurering van de AI.}
\lbentry{dinsdag 20 april 2010}{13:30}{17:45}{04:15}{Deployment van backend en scraper + introductie reguliere expressies in parser.}
\lbentry{woensdag 21 april 2010}{14:39}{15:27}{00:48}{Herstructureren van Perl codebase.}
\lbentry{woensdag 21 april 2010}{15:27}{16:29}{01:02}{Bugfixes in backend, consistent maken van perl scripts.}
\lbentry{woensdag 21 april 2010}{20:48}{22:00}{01:12}{Aanvullen documentatie perl framework.}
\lbentry{donderdag 22 april 2010}{11:00}{12:12}{01:12}{Ontwikkelen van nieuwe functie die quotes aan bepaalde resolutie kan ophalen.}
\lbentry{vrijdag 23 april 2010}{19:47}{20:39}{00:52}{Toevoegen van configuratie van perl framework.}
\lbentry{zaterdag 24 april 2010}{10:00}{11:30}{01:30}{Host spelen.}
\lbentry{zaterdag 24 april 2010}{11:30}{13:00}{01:30}{Testen.}
\lbentry{zaterdag 24 april 2010}{16:30}{17:00}{00:30}{Type fixes en updaten testsuite}
\lbentry{maandag 26 april 2010}{16:00}{17:00}{01:00}{Overzichtspagina versnellen.}
\lbentry{dinsdag 27 april 2010}{13:30}{17:45}{04:15}{Bespreken taakverdeling, werken aan perl framework, debuggen backend.}
\lbentry{dinsdag 27 april 2010}{19:00}{19:45}{00:45}{perl POD fixes en deployment.}
\lbentry{woensdag 28 april 2010}{15:33}{17:33}{02:00}{Werken aan AI.}
\lbentry{woensdag 28 april 2010}{11:00}{12:30}{01:30}{Fixes aan webinterface + introduceren namespaces.}
\lbentry{woensdag 28 april 2010}{21:00}{22:00}{01:00}{Es gespeeld met ASP.NET onder Mono.}
\lbentry{donderdag 29 april 2010}{18:00}{19:00}{01:00}{Verder modulariseren AI.}
\lbentry{donderdag 29 april 2010}{21:45}{22:33}{00:48}{Implementeren Latest Quotes in scraper + maken initi\"le portfolio.}
\lbentry{vrijdag 30 april 2010}{13:30}{17:45}{04:15}{Toevoegen van cache aan de backend.}
\lbentry{vrijdag 30 april 2010}{18:02}{20:13}{02:11}{Caching afwerken.}
\lbentry{zondag 2 mei 2010}{10:30}{13:00}{02:30}{Cache monitor.}
\lbentry{zondag 2 mei 2010}{14:00}{17:47}{03:47}{Meer werk aan cache en backend (hot reload, cache clearing, etc) + documentatie schrijven.}
\lbentry{maandag 3 mei 2010}{11:00}{12:00}{01:00}{Toevoegen van native filter compiler + fixes aan testsuite.}
\lbentry{maandag 3 mei 2010}{18:30}{19:30}{01:00}{Verbeterde toString weergave van filters.}
\lbentry{maandag 3 mei 2010}{19:30}{21:45}{02:15}{Documentatie.}
\lbentry{dinsdag 4 mei 2010}{13:30}{15:15}{01:45}{Documentatie toevoegen aan backend.}
\lbentry{dinsdag 4 mei 2010}{15:15}{17:45}{02:30}{Fixen van bug in scraper + toevoegen van index quotes.}
\lbentry{vrijdag 7 mei 2010}{13:30}{17:45}{04:15}{Afwerken security framework.}
\lbentry{vrijdag 7 mei 2010}{18:00}{19:00}{01:00}{Afwerken security framework.}
\lbentry{zondag 9 mei 2010}{15:30}{16:00}{00:43}{Backend bugfixes.}
\lbentry{maandag 10 mei 2010}{18:00}{18:43}{00:43}{Fixen van autorisatie/authenticatie document + verbeterde error messags.}
\lbentry{dinsdag 11 mei 2010}{13:30}{17:45}{04:15}{Schrijven installer Backend.}
\lbentry{woensdag 12 mei 2010}{18:00}{19:00}{01:00}{Fixen span query.}
\lbentry{dinsdag 13 april 2010}{13:00}{16:00}{03:00}{Overlezen documentatie en fixen van bugs.}
\lbentry{vrijdag 14 mei 2010}{11:00}{14:00}{03:00}{Fixen scraper en invoegen logboeken.}
\lbentry{vrijdag 14 mei 2010}{18:00}{18:30}{00:30}{Fixen namespaces website.}
\lbentry{vrijdag 14 mei 2010}{19:30}{21:00}{01:30}{Verbeteren van verslag, verwijderen van redundante \LaTeX opties.}
\lbentry{vrijdag 14 mei 2010}{21:00}{21:30}{00:30}{Overlezen logboek entries en herschrijven hergebruik sectie van Perl XML-RPC modules.}
\lbentry{vrijdag 14 mei 2010}{21:30}{22:00}{00:30}{Compacter maken van \LaTeX document.}
\lbentry{vrijdag 15 mei 2010}{08:30}{09:10}{00:40}{Uitdiepen van de handleiding en toevoegen van extra screenshots.}
\lbentry{vrijdag 15 mei 2010}{09:45}{10:45}{01:00}{Meer werk aan de protocoldefinitie in het \LaTeX verslag.}
\lbentry{vrijdag 15 mei 2010}{12:00}{15:15}{03:15}{Afwerken van de protocoldefinitie in het \LaTeX verslag.}
\lbentry{zondag 16 mei 2010}{11:30}{12:45}{01:15}{Fixen van hi\"erarchie-foutje in \LaTeX document, verwijderen van redundante methode uit webinterface, en toevoegen van gebruikers en orders aan het Perl framework.}
\lbentry{zondag 16 mei 2010}{15:30}{16:30}{01:00}{Finaliseren deployment-code van het Perl framework, en nog eens code reviewen met behulp van Perl::Critic.}
\lbentry{zondag 16 mei 2010}{13:00}{19:30}{00:30}{Gebruik maken van code floats voor triggers, en omzetten van verbatims uit XML-RPC specificatie naar lstlistings.}
\lbentry{maandag 17 mei 2010}{12:30}{12:50}{00:20}{Toevoegen van plannings-documentatie.}
\lbentry{maandag 17 mei 2010}{18:00}{18:25}{00:25}{Contracten omgezet naar \LaTeX formaat.}
\lbstop{0}
\end{compact}

\section{Dieter Deforce}
\begin{compact}
\lbentry{dinsdag 9 februari 2010}{13:30}{15:30}{02:00}{Presentatie vakoverschrijvend project bijwonen }
\lbentry{dinsdag 9 februari 2010}{15:30}{17:45}{02:15}{Brainstormen over mogelijke onderwerpen en deel van de behoefteanalyse maken }
\lbentry{vrijdag 12 februari 2010}{13:30}{17:45}{04:15}{Verderwerken aan de behoefteanalyse. Voorstellen en verdedigen van projectvoorstel }
\lbentry{dinsdag 16 februari 2010}{13:00}{17:45}{04:45}{Afwerken behoefteanalyse. Maken van de functionele analyse en verdelen van resterende taken onder de groepsleden }
\lbentry{donderdag 18 februari 2010}{09:00}{11:00}{02:00}{Visual Paradigm installeren en uitproberen. Tekenen van use-case diagram met VP. Uitschrijven van verschillende gebruiksgevallen van de desktopapplicatie.}
\lbentry{vrijdag 19 februari 2010}{13:30}{17:45}{04:15}{Bekijken hoe de back-end en XML-RPC protocol kunnen geimplementeerd worden. Starten met prototype van de geanimeerde menubalk voor de desktopapplicatie.}
\lbentry{dinsdag 23 februari 2010}{13:30}{14:30}{01:00}{Presentatie over het schrijven van een bachelor/master-proef door Kathleen Pollefliet }
\lbentry{dinsdag 23 februari 2010}{14:30}{17:45}{03:15}{Bespreken van project met de docenten en vastleggen van de functionaliteit die aanwezig moet zijn in het prototype. Verderwerken aan menubalk.}
\lbentry{woensdag 24 februari 2010}{09:00}{10:30}{01:30}{Prototype van menubalk proberen af te werken. Wegens problemen met de layoutmanagers van Swing, ben ik opnieuw begonnen met eigen paint-code.}
\lbentry{vrijdag 26 februari 2010}{11:00}{12:00}{01:00}{Opzoeken van bestaande biblithoken met controls voor Java Swing. De SwingX-bibliotheek van swinglabs.org heeft gelijkaardige geanimeerde controls aan diegene die we nodig hebben.}
\lbentry{vrijdag 26 februari 2010}{13:30}{17:45}{04:15}{Uitproberen SwingX-bibliotheek. Menucontrol aanpassen naar wens. Starten met het maken van de interfacepanelen van de desktopapplicatie. Beginnen aan de tablemodels voor gebruikers en effecten.}
\lbentry{zondag 28 februari 2010}{11:00}{13:00}{02:00}{Downloaden van Java ME SDK van Samsung en Sun. Proberen beide SDK's op te zetten en een eenvoudig programma werkend te krijgen op mijn GSM.}
\lbentry{maandag 1 maart 2010}{19:45}{22:45}{03:00}{Enkele tutorials over Java ME doorlezen. Een paar interfacepanelen aanmaken en uitproberen op mijn GSM. Uitproberen van eigen paint-code in Java ME.}
\lbentry{dinsdag 2 maart 2010}{13:30}{17:45}{04:15}{Beginnen aan de tablemodels voor de desktopapplicatie. Problemen bespreken met begeleiders. Samen met groepsleden verdere takenverdeling plannen.}
\lbentry{woensdag 3 maart 2010}{16:00}{18:30}{02:30}{Afwerken van tablemodels voor de desktopapplicatie en verder aanpassen van de menubalk. Samenbrengen van menubalk en interfacepanelen in een samenhangende GUI.}
\lbentry{vrijdag 5 maart 2010}{13:30}{17:45}{04:15}{Interfaceklassen aanmaken in ASP.NET applicatie. Overzichtstabel van effecten maken met dummygegevens. Bespreken van mogelijke uitbreidingen van het project met algoritmen (computerspelers mogelijk maken).}
\lbentry{dinsdag 9 maart 2010}{13:30}{17:45}{04:15}{Omzetten van het HTML/CSS design voor de website naar de ASP.NET website.}
\lbentry{donderdag 11 maart 2010}{09:00}{10:00}{01:00}{Herschrijven van enkele use cases voor de desktopapplicatie op vraag van de leerkrachten.}
\lbentry{vrijdag 12 maart 2010}{13:30}{17:45}{04:15}{Aanmaken DataAccess interface. Connectie opzetten met de Oracle-server op school en weergeven van gegevens op de website die uit de databank gehaald worden.}
\lbentry{maandag 15 maart 2010}{13:30}{17:45}{04:15}{Methodes uit de DataAccess interface verder uitschrijven in ADO.NET. Proberen om CSS-opmaak toe te passen op een Gridview in ASP}
\lbentry{maandag 15 maart 2010}{21:00}{22:00}{01:00}{Uitzoeken hoe de nutteloze HTML-tags kunnen verwijderd worden uit de ASP server controls. Dit toepassen op een gridview.}
\lbentry{dinsdag 16 maart 2010}{15:30}{18:30}{03:00}{Connectiestring toevoegen aan ADO.NET om verbinding te kunnen maken met Oracleserver van Laurens. Invullen van enkele dummywaardes in de databank. Verder uitzoeken hoedat je meer controle kunt krijgen over de HTML-output van ASP server controls. Documentatie lezen van de "Css-friendly control adapters" van Micrsoft. Proberen deze adapters te integereren met ons project.}
\lbentry{dinsdag 16 maart 2010}{20:00}{21:30}{01:30}{De CSS-friendly controls blijken toch niet zo "vriendelijk" te zijn. Na verder opzoekwerk blijkt dat je de HTML-output van een ASP control kunt herdefinieren met behulp van een HtmlTextWriter. Deze techniek heb ik uitgeprobeerd op een gridview. Momenteel is de output hiervan nogal gebrekkig, maar het functioneert.}
\lbentry{woensdag 17 maart 2010}{11:00}{12:00}{01:00}{De HtmlTextWriter verder uitwerken zodat de output correct kan werken met de CSS-stylesheet. Ik probeer ook de indentatie in orde te krijgen maar dit blijkt niet mogelijk te zijn}
\lbentry{woensdag 17 maart 2010}{20:00}{22:00}{02:00}{Invoegen van de SiteMapPath aan onze website. Ook deze heeft heel gebrekkige HTML-output en hiervoor heb ik ook een adapter geschreven. Oplossen van een bug met het effectenoverzicht.}
\lbentry{vrijdag 19 maart 2010}{13:30}{17:45}{04:15}{Infosessie over de masterproef. Aanmaken van de detailpagina van een effect. Sitemappath aanpassen zodat deze ook pagina's kan tonen die niet in de SiteMapDataSource vermeld staan.}
\lbentry{zondag 21 maart 2010}{11:00}{12:30}{01:30}{Database query schrijven in ADO.NET om de nodige detailinformatie van een effect op te vragen. Oplossen van probleem met het opvullen van de parameters van deze query (bepaalde variabelenamen zijn gereserveerde woorden bij de Oracle provider)}
\lbentry{zondag 21 maart 2010}{15:00}{18:00}{03:00}{Overzichtspagina van effecten aanpassen zodat de percentages correct ingekleurd worden. Herschikken van de kolommen en toevoegen van enkele gegevens aan de tabel. Opmaak van de detailpagina aanpassen. Extra informatie toevoegen voor een effect.}
\lbentry{dinsdag 23 maart 2010}{13:30}{17:45}{04:15}{Voorbereiden van presentatie van de demo, er waren enkele problemen met het verbinden met de databank. Presentatie geven. Bespreken wat er nog allemaal gedaan moet worden en taken verdelen.}
\lbentry{donderdag 25 maart 2010}{09:00}{10:00}{01:00}{Verslag doornemen en fouten aanduiden.}
\lbentry{vrijdag 26 maart 2010}{10:30}{11:30}{01:00}{\LaTeX en TEXnicCenter installeren en bekijken hoe \LaTeX-documenten zijn opgebouwd.}
\lbentry{vrijdag 26 maart 2010}{13:30}{17:45}{04:15}{Backend uitproberen. Code van backend lezen en proberen te begrijpen. Werking van XML-RPC bekijken en uitproberen. Schrijven van een JUnit test voor de XML-RPC laag.}
\lbentry{zaterdag 27 maart 2010}{09:00}{11:00}{02:00}{Verslag verder lezen en nog enkele fouten verbeteren. Fouten aanpassen in \LaTeX-editor en commiten.}
\lbentry{dinsdag 30 maart 2010}{13:30}{17:45}{04:15}{Bedrijfsbezoek KBC}
\lbentry{donderdag 1 april 2010}{20:00}{22:20}{02:20}{Login pagina toevoegen. Aanpassingen van de databank doorvoeren in ADO.NET. ASP website aanpassen om nu overal ISIN nummers te gebruiken als identificatie van effecten.}
\lbentry{vrijdag 2 april 2010}{13:30}{17:45}{04:15}{Enkele problemen met de Loginpagina oplossen. Binary van XML-RPC.NET importeren in het project. Dummy membershipprovider aanmaken en bekijken hoe deze in elkaar zit.}
\lbentry{zondag 4 april 2010}{20:00}{22:00}{02:00}{Proberen een verbinding op te zetten met de back-end via XML-RPC.NET. Jammergenoeg zit er een ontwerpfout in de back-end waardoor bijna alle functies fouten geven: XML-RPC ondersteunt geen nullwaardes en hier wordt nergens rekening mee gehouden. Teamleden op de hoogte stellen van probleem en wachten op een oplossing.}
\lbentry{maandag 5 april 2010}{16:00}{17:00}{01:00}{Klassenstructuur voor de XML-RPC datalaag maken in Visual Studio.}
\lbentry{dinsdag 6 april 2010}{11:00}{12:00}{01:00}{Huidige datalaag herschikken om performantie te verbeteren. Andere code ook refactoren.}
\lbentry{dinsdag 6 april 2010}{15:00}{18:30}{03:30}{Verder refactoren en gemaakte wijzigingen in de structuur doorvoeren in de rest van het project. Performantieproblemen van ADO.NET laag oplossen (aangezien de huidige structuur ook ongeschikt is voor XML-RPC). Handlerklassen maken die de methodes van XML-RPC vertalen naar C\# methodes.}
\lbentry{woensdag 7 april 2010}{10:00}{11:30}{01:30}{Factory maken voor de datalaag zodat makkelijk wisselen tussen ADO en XML-RPC mogelijk is. Interfaces van de datalaag uitschrijven voor XML-RPC. De nodige methodes om het effectenoverzicht te maken werden eerst geschreven. Hier komen ook enkele bugs in de back-end naar boven en deze moet ik ook oplossen.}
\lbentry{donderdag 8 april 2010}{14:30}{18:30}{04:00}{Constructors schrijven die objecten kunnen maken uit XmlRpcStructs. User klasse schrijven en de nodige methodes die betrekking hebben op inloggen en registreren schrijven. MembershipProvider uitschrijven en uittesten met behulp van een CreateUser-control van ASP. Email veld toevoegen aan een User in backend en databank. }
\lbentry{vrijdag 9 april 2010}{20:30}{22:00}{01:30}{Login mogelijk maken via backend (methode in backend maken die validatie mogelijk maakt). Fouten in paswoordencryptie zoeken samen met Thijs.}
\lbentry{zaterdag 10 april 2010}{10:00}{12:00}{02:00}{Transactiepagina maken. IIS installeren en configureren op laptop en desktop.}
\lbentry{zaterdag 10 april 2010}{14:00}{18:00}{04:00}{Enkele problemen met de MembershipProvider oplossen Website proberen te deployen. Wegens het ontbreken van een 64-bit Oracle Data Provider werkt deze enkel op mijn 32-bit laptop, maar niet op mijn 64-bit desktop. SMTP server opstellen binnen IIS zodat de gebruiker een email ontvangt wanneer hij zich registreert. Omdat de SMTP poort door de meeste providers geblokkeerd wordt is het eenvoudiger om gewoon de SMTP server van de provider/school te gebruiken.}
\lbentry{maandag 12 april 2010}{09:00}{12:00}{03:00}{Status van het project bespreken met groepsleden. Vooruitgang van de verschillende componenten bekijken. Enkele bugs oplossen die nog naar boven komen in de website. Log4NET proberen te configureren}
\lbentry{maandag 12 april 2010}{12:30}{14:00}{01:30}{Log4NET doen loggen naar de Syslog van be03.kapti.com.}
\lbentry{maandag 19 april 2010}{20:00}{23:30}{03:30}{Website klaarmaken voor de deadline van dinsdag: - Verkooporders mogelijk maken - 'Forgot password'-functie toevoegen - CreateUser-wizard vervangen door eigen formulier - Overview-pagina voor de gebruiker maken met zijn belangrijkste informatie en een formulier toevoegen om zijn accountgegevens up te daten - Gegevens portfoliopagina uitbreiden}
\lbentry{dinsdag 20 april 2010}{12:00}{13:15}{01:15}{Tekst en uitleg typen bij de verschillende pagina's van de website. Helppagina toevoegen met wat info over het project}
\lbentry{dinsdag 20 april 2010}{13:30}{17:30}{04:00}{Wijzigingen in CSS om weergave te verbeteren Sell order pagina uitbreiden Zoekfunctie implementeren samen met Laurens en Tim Problemen met javascript oplossen. Bugs in de orderpagina's oplossen en controleren ofdat orders correct worden omgezet in transacties}
\lbentry{zaterdag 24 april 2010}{10:00}{13:30}{03:30}{Projecten van andere groepen uitproberen en hosten opendeurdag}
\lbentry{zondag 25 april 2010}{18:00}{19:30}{01:30}{Internationalisering in ASP.NET bekijken. Uitproberen op een pagina}
\lbentry{maandag 26 april 2010}{20:00}{22:15}{02:15}{Internationalisering doorvoeren op de andere pagina's van de website en tegelijkertijd de vertaling naar het nederlands doen. Problemen met de links proberen op te lossen.}
\lbentry{dinsdag 27 april 2010}{08:15}{09:00}{00:45}{Nog wat verderwerken aan vertaling en werk tot nu toe committen.}
\lbentry{dinsdag 27 april 2010}{13:30}{17:45}{04:15}{Bespreken met groepsleden wat er nog allemaal moet gedaan worden. Laatste pagina's internationaliseren en vertalen. Uitzoeken hoe de sitemap moet vertaald worden. Pagina's overlopen en kijken of er geen tekst ontbreekt en de kapotte links fixen. Zoeken hoe de ASP.NET website vanop afstand kan herstart worden. Implemantie van herstartpagina}
\lbentry{woensdag 28 april 2010}{16:00}{18:00}{02:00}{Logging toevoegen op verschillende plaatsen in de ASP.NET website Nog enkele bugs op verschillende pagina's oplossen en fouten in de vertaling verbeteren.}
\lbentry{vrijdag 30 april 2010}{13:30}{17:45}{04:15}{Begin van het puntenbeheer. Aanmaken van de pointmanager en bekijken van de SPClient die toegang geeft tot de backend. Uitproberen van enkele aanvragen via SPClient en bugs oplossen die hierbij naar boven komen.}
\lbentry{zondag 2 mei 2010}{16:00}{19:00}{03:00}{Verder uitwerken van de pointsmanager. Proberen de winst van spelers terug te rekenen aan de hand van de transacties en hun portfolio. Na een tijdje blijkt dat dit toch niet zo eenvoudig is en bespreek ik met de anderen ofdat we beter niet een extra tabel maken in de databank met de cashposities van de spelers. Uiteindelijk lukt het toch nog om betrouwbaar de winst van de voorbije dag te berekenen (alhoewel het weinig performant is).}
\lbentry{dinsdag 4 mei 2010}{13:30}{17:45}{04:15}{Controleren ofdat de winstberekening nog altijd correct is ook als er complexe koop/verkooporders op dezelfde dag plaatsvinden. Structuur van pointsmanager verbeteren en enkele fouten oplossen in berekeningen. Enkele bugs oplossen in backend, website en SPClient}
\lbstop{0}
\end{compact}

\section{Laurens Van Acker}
\begin{compact}
\lbentry{dinsdag 9 februari 2010}{13:30}{15:30}{02:00}{Project omschrijving, groepsverdeling en vragen}
\lbentry{dinsdag 9 februari 2010}{15:45}{17:45}{02:00}{Kennismaking, brainstorming, opzoeken datafeeds}
\lbentry{vrijdag 12 februari 2010}{18:00}{18:15}{00:15}{Bewerken behoefteanalyse}
\lbentry{zaterdag 13 februari 2010}{17:00}{18:15}{01:15}{Installatie database, ftp hosting en crontab, aanmaken database}
\lbentry{zaterdag 13 februari 2010}{18:15}{18:30}{00:15}{Opstellen werkuren fiche}
\lbentry{dinsdag 16 februari 2010}{13:10}{18:00}{04:50}{GUI uittekenen, databasestructuur en klassenstructuur bepalen}
\lbentry{vrijdag 12 februari 2010}{13:30}{17:45}{04:15}{Eerste evalutie, bespreking en opstellen van analyse}
\lbentry{donderdag 18 februari 2010}{12:00}{12:30}{00:30}{Mookup website}
\lbentry{donderdag 18 februari 2010}{20:20}{00:00}{03:40}{Use cases website: eerste draft opstellen}
\lbentry{vrijdag 19 februari 2010}{13:30}{17:45}{04:15}{Bespreking en afwerking van het backend-protocol.}
\lbentry{zondag 21 februari 2010}{23:00}{23:15}{00:15}{Herwerken tekst}
\lbentry{maandag 22 februari 2010}{19:00}{19:30}{00:30}{Nalezen tekst realisatie en ontwerp.tex}
\lbentry{dinsdag 23 februari 2010}{13:30}{14:30}{01:00}{Marleen Polfliet: Tips bij het schrijven van je Masterproef}
\lbentry{dinsdag 23 februari 2010}{14:30}{17:45}{03:15}{Verdere afwerking protocol, bespreking aangemaakte documenten, evalutatie bij de jury}
\lbentry{dinsdag 23 februari 2010}{21:00}{23:30}{02:30}{Nalezen en herwerken tekst realisatie en ontwerp.tex}
\lbentry{donderdag 25 februari 2010}{14:00}{14:15}{00:15}{Installatie Oracle}
\lbentry{donderdag 25 februari 2010}{19:30}{20:00}{00:30}{Opstellen contract}
\lbentry{vrijdag 26 februari 2010}{11:00}{12:00}{01:00}{Opzoeken bestaande bibliotheken met controls voor Java Swing zoals de SwingX bibliotheek, korte bespreking, onderzoek bibliotheek Plot}
\lbentry{vrijdag 26 februari 2010}{13:30}{17:45}{04:15}{Korte bespreking status, Programmatie Flot statistiekenmodule voor de website, openen poorten FireWall}
\lbentry{dinsdag 2 maart 2010}{13:30}{17:45}{04:15}{Implementatie design, grafiek bibiotheek, evalutatie, }
\lbentry{dinsdag 2 maart 2010}{19:00}{20:00}{01:00}{Implementatie design}
\lbentry{woensdag 3 maart 2010}{17:20}{18:20}{01:00}{Aanmaken en schrijven modules.tex}
\lbentry{donderdag 4 maart 2010}{20:45}{23:00}{02:15}{Installatie server, verder afwerken modules.tex, verder werken design, installatie Visual Studio}
\lbentry{vrijdag 5 maart 2010}{11:00}{12:15}{01:15}{Verder werken design template}
\lbentry{vrijdag 5 maart 2010}{13:30}{17:45}{04:15}{Design template, configureren Oracle server, samen overlopen}
\lbentry{zaterdag 6 maart 2010}{19:00}{21:00}{02:00}{Design}
\lbentry{zondag 7 maart 2010}{14:50}{15:30}{00:40}{Eerste visie van finaal contract opgesteld}
\lbentry{dinsdag 9 maart 2010}{13:30}{17:30}{04:00}{Bespreking, asp website js fouten, oracle configuratie, }
\lbentry{vrijdag 12 maart 2010}{13:30}{17:30}{04:00}{Bespreking, aanvullen en bewerken use cases, filterdemo}
\lbentry{vrijdag 12 maart 2010}{17:00}{17:30}{00:30}{Opzetten server thuis}
\lbentry{zondag 14 maart 2010}{13:00}{16:00}{03:00}{Tussenlaag DOA en XML-RPC een aanpassingen DOA}
\lbentry{donderdag 18 maart 2010}{21:45}{00:00}{02:15}{Javascript grafieken verder werken}
\lbentry{vrijdag 19 maart 2010}{10:30}{12:45}{02:15}{Aanpassingen in de tussenlaag, omzetten van String objecten die de XML-RPC krijgt naar Filter objecten}
\lbentry{vrijdag 19 maart 2010}{13:30}{17:45}{04:15}{Infosessie masterproef, invullen peer-evalution, aanvullen van condities in filter, testen van filter en oplossen van foutjes in filter}
\lbentry{vrijdag 19 maart 2010}{20:00}{21:30}{01:30}{Filter condities toevoegen}
\lbentry{zondag 21 maart 2010}{21:00}{23:00}{02:00}{Voorbereiden demosessie, aanmaken slided}
\lbentry{dinsdag 23 maart 2010}{00:00}{00:45}{00:45}{Wijzingen in presentatie, afdrukken, etc..}
\lbentry{dinsdag 23 maart 2010}{13:30}{17:45}{04:15}{Overlopen presentatie, geven van presentatie, groepsvergadering, bezwaren onderzoeken en samenzetten}
\lbentry{woensdag 24 maart 2010}{21:20}{23:10}{01:50}{Projecten samenvoegen, herschrijven en aanpassen bezwaren}
\lbentry{donderdag 25 maart 2010}{11:00}{11:15}{00:15}{Presentatie in wiki toevoegen en oplossen afhankelijkheidsprobleem van een bibliotheek}
\lbentry{vrijdag 26 maart 2010}{13:45}{17:45}{04:00}{Vergadering, aanpassen exceptionclassen}
\lbentry{zaterdag 27 maart 2010}{15:00}{23:00}{08:00}{Excepties afwerken, hoofdletterongevoeligdheid ingebouwd, aanpassingen realisatie.tex, grafieken object ge\"orienteerd gemaakt en stukken herschreven, grafieken min max functie in alle events, grafieken extra knoppen, etc..}
\lbentry{donderdag 8 april 2010}{20:15}{23:30}{03:15}{Opzetten Tomcat server op arch linux (windows xp wou niet lukken)}
\lbentry{dinsdag 30 maart 2010}{13:30}{17:45}{04:15}{Bedrijfsbezoek KBC}
\lbentry{vrijdag 2 april 2010}{14:00}{17:45}{03:45}{XML-RPC.NET onderzoek en opbouwen ASP.NET interfaces}
\lbentry{vrijdag 9 april 2010}{13:00}{18:00}{05:00}{Herwerken knoppen en functionaliteit grafiek, herschrijven min en max functies, meerdere refferenties tonen, etc..}
\lbentry{maandag 12 april 2010}{16:00}{22:00}{06:00}{Referentiegrafiek, volumegrafiek onderaan en koppeling ertussen, uitdelen van opties en grafieken in verschillende modules met gemeenschappelijke delen}
\lbentry{donderdag 15 april 2010}{19:00}{23:00}{04:00}{Aanpassen ASP.NET deel zodat quotes voor de grafieken opgehaald kunnen worden en onderzoeken en testen JSON service in ASP.NET}
\lbentry{vrijdag 16 april 2010}{10:00}{12:00}{02:00}{Opzetten ASP.NET omgeving, testen deployen ervan, verderwerken grafieken JSON AJAX ophalen van data bij het triggeren van events en het tekenen daarvan}
\lbentry{vrijdag 16 april 2010}{13:30}{21:00}{07:30}{Opzetten ASP.NET omgeving, testen deployen ervan, verderwerken grafieken JSON AJAX ophalen van data bij het triggeren van events en het tekenen daarvan}
\lbentry{zondag 18 april 2010}{14:00}{17:00}{03:00}{Volume ophalen in grafieken, min max functies is ASP en de backend, integratie detail pagina}
\lbentry{zondag 18 april 2010}{22:15}{23:15}{01:00}{Start transactionmanager}
\lbentry{maandag 19 april 2010}{18:45}{19:15}{00:30}{Structuur transactionmanager}
\lbentry{dinsdag 20 april 2010}{13:30}{17:45}{04:15}{Installatie IIS op schoolserver, installatie Oracle connector op thuisserver en schoolserver, deployen van ASP op beide, schrijven documentatie grafieken, onderzoek submit knop searchfield in ASP}
\lbentry{donderdag 22 april 2010}{00:00}{00:40}{00:40}{Documentatie omzetten naar \LaTex, en ASP.NET onderzoeken hoe Javascript variabelen invullen met ASP.NET}
\lbentry{vrijdag 23 april 2010}{23:00}{00:00}{01:00}{Zorgen dat op elke ASP pagina de juiste grafiek weergegeven wordt}
\lbentry{zaterdag 24 april 2010}{00:00}{02:30}{02:30}{Schrijven handleiding en inpassen nieuw databankschema en installatie en compilatie van \LaTeX handleiding en analyse en afdrukken van dit alles}
\lbentry{vrijdag 23 april 2010}{14:00}{17:00}{03:00}{Presentatie masterproeven}
\lbentry{zaterdag 24 april 2010}{08:30}{09:00}{00:30}{Aanpassen beschrijvingen handleiding en vervangen class diagram}
\lbentry{maandag 26 april 2010}{12:00}{13:00}{01:00}{Documenteren Transaction Manager}
\lbentry{dinsdag 27 april 2010}{13:30}{17:45}{04:15}{Verder ontwikkelen transaction manager, port forwarding, peer assignment}
\lbentry{vrijdag 30 april 2010}{13:30}{17:45}{04:15}{Uitbreiden transaction manager met geavanceerde orders en verder debuggen ervan}
\lbentry{zondag 2 mei 2010}{14:00}{20:00}{06:00}{Uitbreiden transaction manager met geavanceerde orders en verder debuggen ervan, testing cache manager}
\lbentry{maandag 3 mei 2010}{11:10}{12:45}{01:35}{Toevoegen secondairy limit op orders in alle lagen, bracket limit orders}
\lbentry{maandag 3 mei 2010}{18:30}{20:15}{01:45}{Bracket Limit en Trailing stop manager}
\lbentry{dinsdag 4 mei 2010}{13:30}{17:45}{04:15}{Herwerken documentatie SPClient en TransactionManager en aanpassen ASP site: ordermodule}
\lbentry{vrijdag 7 mei 2010}{13:30}{17:45}{04:15}{Aanpassen ASP site ordermodule}
\lbentry{zaterdag 8 mei 2010}{16:00}{00:00}{08:00}{Herschrijven en optimalisatie grafiekenfuncties, webservice, verkoopmodule ASP, overlay op grafief voor refferentie}
\lbentry{dinsdag 11 mei 2010}{13:30}{17:45}{04:15}{Dynamische vertalingen grafieken + Reference menu grafieken}
\lbentry{donderdag 13 mei 2010}{11:10}{16:30}{05:20}{Aanpassingen website: Index ophalen en interfaces toegevoegd Grafieken refentiemenu aanpassingen en weghalen deprecated functions. Index objecten waren nodig voor de koppeling tussen ISIN en naam van een index bij een ref}
\lbstop{0}
\end{compact}

\section{Thijs Walcarius}
\begin{compact}
\lbentry{dinsdag 9 februari 2010}{13:30}{15:30}{02:00}{Briefing over VOP, groepsverdeling, etc}
\lbentry{dinsdag 9 februari 2010}{15:45}{17:45}{02:00}{Kennismaking, brainstorming, verkenning beschikbare databronnen, opstellen draft behoeftenanalyse}
\lbentry{vrijdag 12 februari 2010}{13:30}{15:30}{02:00}{Verdere uitwerking van het voorstel van het beursspel}
\lbentry{zondag 14 februari 2010}{09:00}{10:30}{01:30}{TexNicCenter en Miktex installeren, \LaTeX handleiding doorlezen. Omzetten analyse in formaat van hogeschool}
\lbentry{dinsdag 16 februari 2010}{13:00}{18:00}{05:00}{Bijeenkomst met andere groepsleden om technische aspecten van programma te beschrijven: uittekenen GUI, databasestructuur, klassenstructuur, etc}
\lbentry{woensdag 17 februari 2010}{08:15}{12:15}{04:00}{Aanvullen van documentatie op basis van bijeenkomst gisteren: uitzoeken geschikte software voor tekenen klassendiagrammen, herstructureren analysedocument, opzoeken informatie over \LaTeX}
\lbentry{woensdag 17 februari 2010}{19:30}{20:00}{00:30}{Vergelijken documenten Tim en ik om na te gaan of afstemming mogelijk is}
\lbentry{vrijdag 19 februari 2010}{13:30}{17:45}{04:15}{Bespreken gemaakte documentatie, bespreken van implementatie backend en XML-RPC. Mogelijkheden hergebruik voor XML-RPC onderzoeken. Aanmaken van stubs voor businessobjects}
\lbentry{dinsdag 23 februari 2010}{13:30}{17:45}{04:15}{Integreren netbeansproject in SVN, wachten op gesprek met docenten, implementeren van interfaces mbv tijdelijke MySQL-database}
\lbentry{dinsdag 23 februari 2010}{17:45}{19:00}{01:15}{Kijken naar mogelijkheden van gelijktijdige databasetoegang dmv connection pooling zoals mogelijk is in JDBC 3.0 (mbv JDNI)}
\lbentry{vrijdag 26 februari 2010}{13:30}{17:45}{04:15}{Programmeren van business objects, uitzoeken van beste manier om deze met database te laten communiceren. Data Access Objects bekijken, verveeld zijn dat we geen ORM-mappers kunnen gebruiken.}
\lbentry{zaterdag 27 februari 2010}{10:30}{12:00}{01:30}{Implementeren van data access objects voor Exchanges, Securities en SharePrices, zoeken naar bruikbaar alternatief voor Connection Pooling.}
\lbentry{zaterdag 27 februari 2010}{13:00}{16:00}{03:00}{(Her)Installeren VM met Oracle XE, configureren en aanmaken tabellen}
\lbentry{zaterdag 27 februari 2010}{16:15}{18:15}{02:00}{Schrijven van JUnit-tests voor Data Access Objects, testen en corrigeren van fouten}
\lbentry{zaterdag 27 februari 2010}{22:45}{23:30}{00:45}{Opruimen van geschreven rommel, generiek maken van aanspreken DAO}
\lbentry{dinsdag 2 maart 2010}{13:30}{17:45}{04:15}{Inrichten van 1 schoolserver, verder programmeren aan DAO, Sequences in Oracle bekijken}
\lbentry{woensdag 3 maart 2010}{08:45}{10:45}{02:00}{DAO-laag verder afwerken}
\lbentry{vrijdag 5 maart 2010}{11:00}{12:00}{01:00}{Bekijken van templates in ASP.NET}
\lbentry{vrijdag 5 maart 2010}{13:30}{17:45}{04:15}{Experimenteren met ADO.Net, Oracle-installatie proberen porten naar nieuwe server}
\lbentry{maandag 8 maart 2010}{18:00}{19:00}{01:00}{Documentatie schrijven over DAO, verder zoeken op porten van Oracle-installatie}
\lbentry{dinsdag 9 maart 2010}{13:30}{17:45}{04:15}{Effectief overzetten van database uit VM naar echte server. Experimenteren met XML-RPC-client voor JavaSE}
\lbentry{donderdag 11 maart 2010}{10:30}{12:00}{01:30}{Experimenteren met XML-RPC-client voor Java ME en werking Java ME in het geheel}
\lbentry{vrijdag 12 maart 2010}{13:30}{17:45}{04:15}{XML-RPC-server analyseren, functies schrijven om XML-RPC-requests af te handelen}
\lbentry{maandag 15 maart 2010}{20:30}{22:30}{02:00}{SwingX-componenten bekijken en documentatie doorlezen. Uittesten van mogelijkheden}
\lbentry{dinsdag 16 maart 2010}{07:30}{08:00}{00:30}{Eerste aanzet voor desktopapplicatie}
\lbentry{dinsdag 16 maart 2010}{13:30}{17:45}{04:15}{Ontwerpen van gridbag-layout voor desktopapplicatie. Begin van XML-RPC-client}
\lbentry{woensdag 17 maart 2010}{09:00}{12:00}{03:00}{Ontwerpen menu van Administratie-client. Securities-weergave maken. Licht herwerken structuur desktopapplicatie.}
\lbentry{woensdag 17 maart 2010}{20:30}{22:00}{01:30}{Proberen doorgronden van werking XML-server en filters..}
\lbentry{donderdag 18 maart 2010}{09:00}{12:00}{03:00}{Uitzoeken van werking CPAN, downloaden en installeren van Oracle Instantclient, proberen aan de praat krijgen van DBD::Oracle-module, schrijven van import-scriptje voor historische data van aandelen}
\lbentry{vrijdag 19 maart 2010}{10:30}{12:45}{02:15}{Verder afwerken import-scriptje, proberen extra omgeving thuis op te zetten om importeren uit te voeren, klein beetje schrijven aan use-cases}
\lbentry{vrijdag 19 maart 2010}{13:30}{17:45}{04:15}{Infosessie masterproef, invullen peer-evaluation, aanpassen databank, kleine wijzigingen aan desktopapplicatie}
\lbentry{vrijdag 19 maart 2010}{19:00}{20:15}{01:15}{En masse binnenhalen van historische data van enkele aandelen NASDAQ, Euronext amsterdam en Euronext Brussel}
\lbentry{zaterdag 20 maart 2010}{11:00}{13:00}{02:00}{Werking van backend bestuderen, proberen JUnit-tests te schrijven}
\lbentry{zaterdag 20 maart 2010}{16:00}{18:30}{02:30}{Verder JUnit-tests schrijven, werken aan desktopapplicatie, foutjes oplossen}
\lbentry{dinsdag 23 maart 2010}{13:30}{17:45}{04:15}{Presentatie demo. Verdere uitstippeling van agenda voor komende weken}
\lbentry{vrijdag 26 maart 2010}{13:30}{17:45}{04:15}{Werken aan XML-RPC-backend, testen van aanvragen met JUnit}
\lbentry{dinsdag 30 maart 2010}{13:30}{17:45}{04:15}{Bedrijfsbezoek KBC}
\lbentry{vrijdag 2 april 2010}{13:30}{17:45}{04:15}{Debuggen van XML-RPC-backend: oplossen fouten in User-gedeelte}
\lbentry{dinsdag 6 april 2010}{10:00}{12:15}{02:15}{Herschikken van code desktopapplicatie: flexibilisering en opkuis}
\lbentry{dinsdag 6 april 2010}{15:00}{17:30}{02:30}{Uitbreiden van effectenbeheer-gedeelte desktopapplicatie}
\lbentry{donderdag 8 april 2010}{09:00}{11:30}{02:30}{Geen bericht.}
\lbentry{zondag 11 april 2010}{14:00}{17:00}{03:00}{Afwerken componentenbeheer bij desktopapplicatie}
\lbentry{maandag 12 april 2010}{09:00}{15:00}{06:00}{Begin van gebruikersbeheer in desktopapplicatie: bijwerken van Securitiesbeheer, en omzetten naar gebruikersbeheer, debuggen java-xml-rpc-client}
\lbentry{dinsdag 13 april 2010}{13:30}{16:45}{03:15}{Geen bericht}
\lbentry{woensdag 14 april 2010}{14:00}{17:00}{03:00}{Vervangen van admin-boolean door rollen in backend, ook bijwerken in desktopapplicatie. Zorgen dat gegenereerde id's worden teruggegeven}
\lbentry{donderdag 15 april 2010}{09:00}{11:00}{02:00}{Verder werken aan gebruikersbeheer in desktopapplicatie - implementeren generieke dialoogvenster voor veranderingen met reden}
\lbentry{donderdag 15 april 2010}{20:00}{22:45}{02:45}{Zoeken op werking triggers, diverse problemen  proberen oplossen}
\lbentry{vrijdag 16 april 2010}{09:00}{11:30}{02:30}{Aanmaken puntentransacties - zoeken op triggers}
\lbentry{vrijdag 16 april 2010}{14:00}{17:00}{03:00}{Gebruikersbeheer verder afwerken, fouten uit backend halen}
\lbentry{zaterdag 17 april 2010}{13:00}{15:30}{02:30}{Herwerken van client-backend naar eleganter systeem}
\lbentry{zaterdag 17 april 2010}{16:00}{20:00}{04:00}{Debuggen backend, herwerken en toevoegen van functionaliteit voor transacties - toevoegen van triggers}
\lbentry{zondag 18 april 2010}{13:30}{16:30}{03:00}{Toevoegen order- en transactionstukken aan Java XML-RPC clientlibrary}
\lbentry{maandag 19 april 2010}{18:15}{19:00}{00:45}{Uittesten van functionaliteit voor ophalen quotes, debuggen}
\lbentry{maandag 19 april 2010}{20:30}{23:45}{03:15}{Eerste functionele versie maken van transactiemanager}
\lbentry{dinsdag 20 april 2010}{13:30}{17:30}{04:00}{Deployen van betaversie en oplossen van bugs}
\lbentry{donderdag 22 april 2010}{19:30}{20:30}{01:00}{Uitsplitsen van Java XML-RPC-client naar aparte library}
\lbentry{zaterdag 24 april 2010}{10:00}{13:30}{03:30}{Testen betaversies en host spelen op opendeurdag}
\lbentry{zondag 25 april 2010}{13:30}{17:30}{04:00}{Herwerken Java XML-RPC-client naar Java ME}
\lbentry{dinsdag 27 april 2010}{13:30}{17:45}{04:15}{Bespreken verdere planning project, uitzoeken werking eID-libraries}
\lbentry{woensdag 28 april 2010}{09:15}{11:15}{02:00}{Werking onderzoeken van java eID-library-wrapper, uitzoeken hoe te integreren in netbeansproject}
\lbentry{woensdag 28 april 2010}{14:00}{15:30}{01:30}{Brielen met eerste implementatie loginscherm, zal opnieuw moeten gedaan worden..}
\lbentry{donderag 29 april 2010}{9:30}{12:00}{02:30}{Begin internationalisatie van desktopapplicatie}
\lbentry{vrijdag 30 april 2010}{13:30}{17:45}{04:15}{Herwerken desktopapplicatie: we maken een menubalk om instellingen toe te laten, maar herwerken daarvoor het menu aan de linkerkant tot acties zodat het ook kan gebruikt worden voor extra items in de menubalk. Uitzoeken van beste manier om instellingen op te slaan}
\lbentry{zaterdag 1 mei 2010}{10:00}{15:00}{05:00}{Toevoegen van rollen aan database om beveiliging mogelijk te maken}
\lbentry{zondag 2 mei 2010}{14:00}{16:00}{02:00}{Onderzoeken van functionaliteit Apache XML-RPC-server op gebied van authenticatie, en hoe ik best mijn eigen hook ervoor maak, uittesten van verschillende mogelijkheden}
\lbentry{dinsdag 4 mei 2010}{13:30}{17:45}{04:15}{Aanpassen van backend en SPClient zodat deze kunnen omgaan met sessies}
\lbentry{donderdag 7 mei 2010}{10:00}{12:00}{02:00}{Bugfixen van vervelende fouten in sessiehandeling, herwerken van SPClient om manier van opslaan authenticatie te verbeteren}
\lbentry{vrijdag 8 mei 2010}{13:30}{14:15}{00:45}{Zoeken naar manier om invalid connecties uit de Apache DB Connection Pool te halen, en onze backend er niet meer over te doen struikelen}
\lbentry{vrijdag 8 mei 2010}{14:15}{17:45}{03:30}{Afwerken van SPClient om hem compatibel te maken met sessies op de backend}
\lbentry{zaterdag 9 mei 2010}{9:00}{12:00}{03:00}{Herwerken van SPClient zodat er ondersteuning komt voor de private-url}
\lbentry{zaterdag 9 mei 2010}{14:00}{18:30}{04:30}{Herwerken loginprocedure van desktopapplicatie zodat meerdere loginmogelijkheden kunnen worden aangeboden. Configureerbaar maken van loginprocedures. Starten aan implementatie van eID-login}
\lbentry{zondag 10 mei 2010}{9:15}{11:45}{02:30}{Functionaliteit aanmaken zodat met de eID kan worden ingelogd op de beheersapplicatie. Om de bestaande beveiliging op de backend niet onderuit te halen is het wel niet zo mooi geworden}
\lbentry{zondag 10 mei 2010}{13:30}{17:00}{03:30}{Java ME-applicatie uitbreiden qua functionaliteit, en wat herstructureren}
\lbentry{maandag 11 mei 2010}{7:45}{8:30}{00:45}{Snel maken van een before trigger}
\lbentry{dinsdag 12 mei 2010}{12:15}{13:15}{01:00}{Afwerken van eID-login, controle van functionaliteit van Preferences}
\lbentry{dinsdag 12 mei 2010}{13:30}{17:45}{04:15}{Aanpassen van kXMLRPC library zodat deze om kan gaan met authenticatie met de backend (er was nog geen funcationaliteit aanwezig om aan Basic Authentication te doen}
\lbentry{donderdag 13 mei 2010}{9:30}{12:00}{02:30}{GSM-applicatie opschonen en proberen afwerken, debuggen van fouten door verkeerde aanvragen naar de backend}
\lbentry{donderdag 13 mei 2010}{15:45}{18:00}{02:15}{Opschonen van administratieprogramma: code wat herwerken, slecht functionerende onderdelen verwijderen}
\lbentry{vrijdag 14 mei 2010}{9:00}{10:00}{01:00}{Maken van betere before trigger om iets nuttigs te controleren..}
\lbentry{vrijdag 14 mei 2010}{10:00}{12:00}{02:00}{Controle functionaliteit administratieprogramma, afwerken internationalisatie}
\lbentry{vrijdag 14 mei 2010}{13:30}{15:30}{02:00}{Verder testen van het administratieprogramma, proberen foutjes te ontdekken en recht te trekken}

\lbstop{0}
\end{compact}

